%!TEX root = Main.tex

%define graphics path:
\graphicspath{{./figures/}}



\setlength{\unitlength}{\textwidth}
%% DEFINE NEW SI UNITS


\DeclareSIUnit{\Wm}{\watt \per \meter \squared}


%Brakets (or square Brackets) 
\DeclareSIUnit{\rB}{\ensuremath{]}}
\DeclareSIUnit{\lB}{\ensuremath{[}}

%Parantheses
\DeclareSIUnit{\rP}{\ensuremath{)}}
\DeclareSIUnit{\lP}{\ensuremath{(}}

\DeclareSIUnit{\div}{\ensuremath{/}}

\sisetup{separate-uncertainty=true}

\sisetup{detect-all=true}
\sisetup{retain-explicit-plus=true}

% good looking units in captions
\newcommand{\capunit}[1]{\si[detect-all=false]{\lB#1\rB}}


% example commands
% \newcommand{\p}[1]{\ensuremath{\text{p}_{#1}}} %percentile
% \newcommand{\sunel}{\ensuremath{\theta_e}\xspace}
% \newcommand{\KZ}{K\&Z\xspace}


% dot for f(.)
\newcommand*{\edot}{\makebox[1ex]{\textbf{$\cdot$}}}%

\newcommand{\hl}[2]{\ensuremath{\text{#1}_{\text{#2}}}}

\newcommand{\gridRes}[2]{\SI{#1}{\degree} $\times$ \SI{#2}{\degree}}

%% COMMANDS


% \fig{(b)}{fig_name}
\newcommand{\fig}[2][]{%
Figure~\ref{fig:#2}\ifthenelse{\isempty{#1}}{}{(#1)}%
}

%\newcommand{\fig}[1]{Figure~\ref{fig:#1}}
%\eqref{} defined by amsmath
\newcommand{\tab}[1]{Table~\ref{tab:#1}}
\newcommand{\sect}[1]{Section~\ref{#1}}
\newcommand{\chap}[1]{Chapter~\ref{#1}}


%% TABLES
\newcolumntype{C}[1]{>{\centering\arraybackslash}p{#1}}
\newcolumntype{L}[1]{>{\raggedright\arraybackslash}p{#1}}

% FIGURES
\newcommand{\singlefig}[2]{%
\begin{figure}[h!tbc]
        \centering
        \includegraphics[width=\textwidth]{#1}
			\caption{#2 \label{fig:#1}}
\end{figure}
}




\newcommand{\doublefig}[4]{%
\begin{figure}[h!tbc]
        \centering
              \begin{minipage}[t]{0.48\textwidth}
                \centering
                \includegraphics[width=\textwidth]{#1}
				\captionof{figure}{#2 \label{fig:#1}}
        \end{minipage}%
		\hfill
\begin{minipage}[t]{0.48\textwidth}
                \centering
                \includegraphics[width=\textwidth]{#3}
				\captionof{figure}{#4 \label{fig:#3}}
        \end{minipage}%
\end{figure}
}
